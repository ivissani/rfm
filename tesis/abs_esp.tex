%!TEX root = tesis.tex

%\begin{center}
%\large \bf \runtitulo
%\end{center}
%\vspace{1cm}
\chapter*{\runtitulo}

En la actualidad el \soft forma parte integral de nuestra vida. Este
interviene  en todo tipo de tareas. Desde el envío de un mensaje de texto o la
realización de una llamada telefónica hasta el sistema de control de una
central nuclear o el sistema de velocidad crucero de un automóvil, utilizan
sistemas basados en \hard y \soft informático.

El análisis de software es un área en la ingeniería de software que en general
se ocupa de la aplicación de métodos formales que permiten garantizar, o al
menos ganar confianza en la ausencia de errores en una pieza de \soft. Existe
una plétora de herramientas dedicadas a esta tarea, cada una de ellas con sus
elecciones particulares de lenguajes y método a aplicar. Entre ellas, una
cantidad no menor se ha volcado al uso de \ssolvers \ots como mecanismo de
bajo nivel que permite explorar eficientemente un espacio finito de modelos.

En el presente trabajo pretendemos implementar un \ssolver paralelo y
distribuido que: \begin{inparaenum}[a)]   \item realice una adecuación y
balanceo de la carga de procesamiento en \emph{run-time} sobre una cantidad y
configuración de equipamiento desconocida, \item realice una adecuación y
balanceo del uso de memoria primaria y secundaria en \emph{run-time}, también
sobre una cantidad y configuración de equipamiento desconocida, y  \item sea
extensible y fácil de parametrizar en todos aquellos aspectos que determinen
el comportamiento del procedimiento de resolución del problema.
\end{inparaenum} Una vez llevada a cabo esta tarea, pretendemos evaluar la
viabilidad de implementar una técnica de aprendizaje basada en análisis de
cláusulas de conflicto acorde a las necesidades de ambientes paralelos y
distribuidos, y su comparación empírica.

\bigskip

\noindent \textbf{Palabras claves:} Software engineering, Verificación and
validación, Sat-solvers, Parallelo y distribuido, Aprendizaje basado en
conflictos.
