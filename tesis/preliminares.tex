%!TEX root = tesis.tex

\newcommand{\disj}[1]{\ensuremath{[#1]}}
\newcommand{\conj}[1]{\ensuremath{\{#1\}}}

\section{Lógica proposicional}
% El origen de la lógica proposcional, o lógica de predicados puede ser rastreado hasta la antigua Grecia. 
La lógica proposicional es la rama de la lógica de estudia los predicados y sus combinaciones.
\begin{definition}
\defname{Predicado} Un predicado es una sentencia que puede ser verdadera o falsa
\end{definition}

La formalización más utilizada de la lógica proposicional es la desarrollada
por el matemático inglés George Boole.  La formalización desarrollada por
Boole es un formalización algebraica

\begin{definition}
\todo{Revisar esta definición}
\defname{Álgebra de Boole} Un álgebra de Boole consiste de 
	\begin{enumerate}
		\item Un conjunto de variables proposicionales $\{p, q, \ldots\}$ que toman valores del conjunto $\{True, False\}$ ($\{1, 0\}$)
		\item Dos operadores binarios $\wedge\ (\cdot)$  y $\vee\ (+)$
		\item Un operador unario $\neg\ (-)$
	\end{enumerate}
	Tal que 
	\begin{itemize}
		\item Si $p$ es una variable proposicional, entonces $p$ y $-p$ son literales
		\item Un literal es una fórmula proposicional
		\item Si $\phi_1$ y $\phi_2$ son fórmulas proposicional entonces $\phi_1 \cdot \phi_2$ es una fórmula, $\phi_1 + \phi_2$ es una fórmula y $-\phi_1$ es una fórmula
	\end{itemize}
	y se cumplen las siguientes propiedades
	\begin{itemize}
		\item $--p = p$
		\item $p + q = q + p$ y $p \cdot q = q \cdot p$ (conmutatividad)
		\item $p + (q + r) = (p + q) + r$ y $p \cdot (q \cdot r) = (p \cdot q) \cdot r$ (asociatividad)
		\item $p + (q \cdot r) = (p + q) \cdot (p + r)$ y $p \cdot (q + r) = (p \cdot q) + (p \cdot r)$ (distributividad)
		\item $p + 0 = p$ y $p \cdot 1 = p$ 
		\item $p + 1 = 1$ y $p \cdot 0 = 0$
		\item $p + -p = 1$ y $p \cdot -p = 0$
		\item $p \cdot (p + q) = p$ y $p + (p \cdot q) = p$ (absorción)
	\end{itemize}
\end{definition}

\begin{definition}
\defname{Valuación} Sea $\varphi$ una fórmula de la lógica proposicional con variables $\{p_1, \ldots, p_n\}$, entonces una valuación de la fórmula $\varphi$ es una asignación de los valores $\{0, 1\}$ a cada una de las variables proposicionales de $\varphi$.
\end{definition}

Por ejemplo, sean $\{v_1, \ldots, v_n\}$ con $v_i \in \{0, 1\}$ y una fórmula proposicional $\varphi$ con variables $\{p_1, \ldots, p_n\}$, entonces $\{p_1 \leftarrow v_1, \ldots, p_n \leftarrow v_n\}$ es una valuación.

\begin{definition}
\defname{Valuación parcial} Una valuación parcial de una fórmula proposicional $\varphi$ es una valuación sobre un subconjunto propio de las variables proposicionales de $\varphi$
\end{definition}

\begin{definition}
\defname{Fórmula satisfacible (\sat)} Una fórmula proposicional $\varphi$ es satisfacible si existe al menos una valuación que haga que la fórmula sea verdadera. Si la fórmula $\varphi$ bajo la valuación $V$ es verdadera, decimos que $V$ satisface a $\varphi$.
\end{definition}

\begin{definition}
\defname{Tautología} Una fórmula $\varphi$ es una tautología si y solo si toda valuación $V$ la satisface.
\end{definition}

\begin{definition}
\defname{Contradicción} Una fórmula $\varphi$ es una contradicción si y solo si ninguna valuación $V$ la satisface.
\end{definition}

\begin{definition}
\defname{Fórmula insatisfacible (\unsat)} Una fórmula $\varphi$ es insatisfacible (o \unsat) si y solo si es una contradicción.
\end{definition}


\subsection{Forma Normal Conjuntiva}

\begin{definition} \defname{Forma Normal Conjuntiva} Una fórmula $\varphi$ de
la lógica proposicional se encuentra en Forma Normal Conjuntiva (\cnf por sus
siglas en inglés) si la misma es una conjunción de cláusulas, donde cada
cláusula es una disyunción de literales. \end{definition}

Por ejemplo, la fórmula $\phi = (p \vee q) \wedge (-p \vee q)$ se encuentra
en \cnf y posee dos cláusulas $C_1 = p \vee q$ y $C_2 = -p \vee q$. Dado que los
conectores lógicos están implícitos cuando una fórmula se encuentra en \cnf
normalmente escribimos $\phi = \conj{\disj{p, q}, \disj{-p, q}}$ de modo equivalente.

\begin{definition}
\defname{Fórmulas equivalentes} Dos fórmulas proposicionales $\varphi_1$ y $\varphi_2$ son equivalentes si y solo si para toda valuación $V$, $V$ satisface a $\varphi_1$ $\Longleftrightarrow$ $V$ satisface a $\varphi_2$.
\end{definition}

\begin{theorem}\label{th:cnfequiv}
Para toda fórmula de la lógica proposicional es posible encontrar una fórmula equivalente que se encuentre en \cnf
\end{theorem}

\section{El problema de la SATisfactiblidad}

El problema de la satisfactibilidad (\emph{SAT problem}) consiste en determinar
si existe alguna asignación de valores de verdad (\true|\false) a cada una de
las variables booleanas que aparecen en una fórmula $\phi$ de la lógica
proposicional de modo que $\phi$ se haga verdadera. En caso de que dicha
asignación exista decimos que $\phi$ es satisfacible (o simplemente \sat) y la
asignación o valuación $\{ p_1 \leftarrow v_1, \ldots, p_n \leftarrow v_n \}$
correspondiente es un \textbf{modelo} de $\phi$. Es importante notar que en caso
de que una fórmula sea \sat podría existir más de una valuación que haga que la
fórmula sea verdadera. Si no existe ninguna valuación que haga verdadera a
$\phi$ decimos que $\phi$ es insatisfacible (o \unsat).

Dado el teorema \ref{th:cnfequiv} el problema \sat se define equivalentemente
como el problema de determinar si una fórmula en \cnf es satisfacible o no.

El problema de determinar si una fórmula $\phi$ dada es \sat o \unsat pertenece
a la clase de problemas \npc\cite{Cook:1971:CTP:800157.805047}. Es decir que no
se conoce un algoritmo cuya complejidad temporal sea polinomial que pueda determinar si una fórmula es \sat
o \unsat. La importancia que reviste el problema SAT para diversas áreas de las
ciencias de la computación, como ser el diseño de circuitos o la verificación
automática, ha impulsado el desarrollo de algoritmos relativamente eficientes
para una clase amplia de problemas conocidos como \emph{SAT solvers}.

\subsection{\emph{SAT Solving}}

Se conoce con el nombre de \emph{SAT solvers} a los algoritmos que permiten
determinar si una fórumla $\phi$ de la lógica proposicional tiene alguna
valuación que la haga verdadera. El algoritmo general de  \emph{SAT solving}
consiste en un algoritmo de \bt que podríamos esquematizar como sigue:

\missingfigure{Algoritmo SAT solving backtracking}

La mayoría de los \emph{SAT solvers} son
variaciones del algoritmo general
DPLL\cite{Davis:1962:MPT:368273.368557} y actúan sobre fórmulas
proposicionales expresadas en Forma Normal Conjuntiva (\cnf).



\begin{definition} 
\defname{Cláusula unitaria} Una cláusula se llama \textbf{unitaria} si
posee un único literal.
\end{definition}
Por ejemplo
en la fórmula $\gamma = \conj{\disj{-p}, \disj{q, r}}$ la clásula $\disj{-p}$ es
unitaria. Cuando una fórmula \cnf posee una cláusula unitaria, el único modo de
lograr que la fórmula completa sea verdadera es asignando el valor de verdad
\true al único literal dentro de la cláusula unitaria.

\subsection{Algoritmo DPLL}

El algoritmo DPLL es un procedimiento 

\begin{lstlisting}[mathescape,language=Pascal,frame=single,numbers=left,caption={Algoritmo
DPLL}] G = F
while G includes a clause C such that |C| $\leq$ 1 do
  if C = $\emptyset$ then return G
  else if C = {v} then G = G|v
end while
while there is a monotone literal in G do
  v = any monotone literal
  G = G|v
end while
return G
\end{lstlisting}

\subsection{Aprendizaje}
\subsubsection{First UIP}
\subsubsection{Recorte de la base de datos}
\subsubsection{Actividad y LBD}

\section{Alloy}

\section{ParAlloy}
\subsection{BEDs}

\section{\emph{SAT Solving} distribuido}

\subsection{Aprendizaje}
