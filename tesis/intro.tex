En la actualidad el \soft forma parte integral de nuestra vida cotidiana. El
mismo interviene en todo tipo de tareas. Desde el envío de un mensaje de texto o
la realización de una llamada telefónica hasta el sistema de control de una
central nuclear, pasando por el control de un ascensor o el sistema de velocidad
crucero de un automóvil, utilizan sistemas basados en \hard y \soft
informático. La variedad, alcance y \textbf{criticidad} de las responsabilidades
asignadas a las piezas de \soft hacen de las mismas un componente de alto
impacto en nuestra realidad actual. 

Por otra parte, la alta complejidad del \soft hace que la construcción del mismo
sea una tarea propensa a errores. El impacto de una falla en un componente de
\soft puede variar desde la imposibilidad de utilizar un artefacto doméstico
hasta una catástofre de magnitudes como la fundición del núcleo de un reactor
nuclear. 

Son estos aspectos los que motivan la necesidad creciente de construir métodos
(formalismos, metodologías, herramientas, etc.) que permitan garantizar la
calidad del \soft en un sentido general. La posibilidad de establecer la
ausencia de errores en un programa radica en la capacidad de establecer
inequívocamente que ese programa cumple una determinada propiedad que exprese
el adecuamiento del mismo al comportamiento esperado.
Por ejemplo si quisiéramos establecer el correcto  funcionamiento ``¿Será cierto que no es
posible encender el microondas si la puerta está abierta?''