%!TEX root = tesis.tex

\newcommand{\disj}[1]{\ensuremath{[#1]}}
\newcommand{\conj}[1]{\ensuremath{\{#1\}}}

\section{El problema de la SATisfactiblidad}

El problema de la satisfactibilidad (\emph{SAT problem}) consiste en determinar
si existe alguna asignación de valores de verdad (\true|\false) a cada una de
las variables booleanas que aparecen en una fórmula $\phi$ de la lógica
proposicional de modo que $\phi$ se haga verdadera. En caso de que dicha
asignación exista decimos que $\phi$ es satisfacible (o simplemente \sat) y la
asignación o valuación $\{ v_1 \leftarrow V_1, \ldots, v_n \leftarrow V_n \}$
correspondiente es un \textbf{modelo} de $\phi$. Es importante notar que en caso
de que una fórmula sea \sat podría existir más de una valuación que haga que la
fórmula sea verdadera. Si no existe ninguna valuación que haga verdadera a
$\phi$ decimos que $\phi$ es insatisfacible (o \unsat).

El problema de determinar si una fórmula $\phi$ dada es \sat o \unsat pertenece
a la clase de problemas \npc\cite{Cook:1971:CTP:800157.805047}. Es decir que no
se conoce un algoritmo cuya complejidad temporal sea polinomial que pueda determinar si una fórmula es \sat
o \unsat. La importancia que reviste el problema SAT para diversas áreas de las
ciencias de la computación, como ser el diseño de circuitos o la verificación
automática, ha impulsado el desarrollo de algoritmos relativamente eficientes
para una clase amplia de problemas conocidos como \emph{SAT solvers}.

\subsection{\emph{SAT Solving}}

Se conoce con el nombre de \emph{SAT solvers} a los algoritmos que permiten
determinar si una fórumla $\phi$ de la lógica proposicional tiene alguna
valuación que la haga verdadera. El algoritmo general de  \emph{SAT solving}
consiste en un algoritmo de \bt que podríamos esquematizar como sigue:

\missingfigure{Algoritmo SAT solving backtracking}

La mayoría de los \emph{SAT solvers} son
variaciones del algoritmo general
DPLL\cite{Davis:1962:MPT:368273.368557} y actúan sobre fórmulas
proposicionales expresadas en Forma Normal Conjuntiva (\cnf).

\subsection{Forma Normal Conjuntiva}

Decimos que una fórmula de la lógica proposicional se encuentra en Forma Normal
Conjuntiva (\cnf por sus siglas en inglés) si la misma es una conjunción de
cláusulas, donde cada cláusula es una disyunción de literales; un literal es una
variable $v$ o su negación $-v$.

Por ejemplo, la fórmula $\phi = (p \vee q) \wedge (-p \vee q)$ se encuentra
en \cnf y posee dos cláusulas $C_1 = p \vee q$ y $C_2 = -p \vee q$. Dado que los
conectores lógicos están implícitos cuando una fórmula se encuentra en \cnf
normalmente escribimos $\phi = \conj{\disj{p, q}, \disj{-p, q}}$ de modo equivalente.

\begin{definition} 
\defname{Cláusula unitaria} Una cláusula se llama \textbf{unitaria} si
posee un único literal.
\end{definition}
Por ejemplo
en la fórmula $\gamma = \conj{\disj{-p}, \disj{q, r}}$ la clásula $\disj{-p}$ es
unitaria. Cuando una fórmula \cnf posee una cláusula unitaria, el único modo de
lograr que la fórmula completa sea verdadera es asignando el valor de verdad
\true al único literal dentro de la cláusula unitaria.

\subsection{Algoritmo DPLL}

El algoritmo DPLL es un procedimiento 

\begin{lstlisting}[mathescape,language=Pascal,frame=single,numbers=left,caption={Algoritmo
DPLL}] G = F
while G includes a clause C such that |C| $\leq$ 1 do
  if C = $\emptyset$ then return G
  else if C = {v} then G = G|v
end while
while there is a monotone literal in G do
  v = any monotone literal
  G = G|v
end while
return G
\end{lstlisting}

\subsection{Aprendizaje}
\subsubsection{First UIP}
\subsubsection{Recorte de la base de datos}
\subsubsection{Actividad y LBD}

\section{Alloy}

\section{ParAlloy}
\subsection{BEDs}

\section{\emph{SAT Solving} distribuido}

\subsection{Aprendizaje}