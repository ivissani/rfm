\documentclass[a4paper, 11pt, twoside]{tesis}

\usepackage[utf8]{inputenc}
\usepackage[spanish]{babel}
\usepackage{graphicx}
\usepackage{xspace}
\usepackage{cite}
\usepackage{listings}
\usepackage[colorinlistoftodos, shadow]{todonotes}
% \usepackage[disable]{todonotes}

\begin{document}

%%%% CARATULA
\def\autor{Ignacio Vissani}
\def\titulo{Acelerando ParAlloy mediante la reutilización de cláusulas
aprendidas} 
\def\runtitulo{Acelerando ParAlloy mediante la reutilización de cláusulas
aprendidas}
\def\runtitle{Speeding up ParAlloy reusing learnt clauses}
\def\director{Carlos Gustavo López Pombo}
\def\codirector{Nicolás Leandro Rosner}
\def\lugar{Buenos Aires, 2012}
\input{caratula}

%%%% ABSTRACTS, AGRADECIMIENTOS Y DEDICATORIA
\frontmatter
\pagestyle{empty}
%!TEX root = tesis.tex

%\begin{center}
%\large \bf \runtitulo
%\end{center}
%\vspace{1cm}
\chapter*{\runtitulo}

En la actualidad el \soft forma parte integral de nuestra vida. Este
interviene  en todo tipo de tareas. Desde el envío de un mensaje de texto o la
realización de una llamada telefónica hasta el sistema de control de una
central nuclear o el sistema de velocidad crucero de un automóvil, utilizan
sistemas basados en \hard y \soft informático.

El análisis de software es un área en la ingeniería de software que en general
se ocupa de la aplicación de métodos formales que permiten garantizar, o al
menos ganar confianza en la ausencia de errores en una pieza de \soft. Existe
una plétora de herramientas dedicadas a esta tarea, cada una de ellas con sus
elecciones particulares de lenguajes y método a aplicar. Entre ellas, una
cantidad no menor se ha volcado al uso de \ssolvers \ots como mecanismo de
bajo nivel que permite explorar eficientemente un espacio finito de modelos.

En el presente trabajo pretendemos implementar un \ssolver paralelo y
distribuido que: \begin{inparaenum}[a)]   \item realice una adecuación y
balanceo de la carga de procesamiento en \emph{run-time} sobre una cantidad y
configuración de equipamiento desconocida, \item realice una adecuación y
balanceo del uso de memoria primaria y secundaria en \emph{run-time}, también
sobre una cantidad y configuración de equipamiento desconocida, y  \item sea
extensible y fácil de parametrizar en todos aquellos aspectos que determinen
el comportamiento del procedimiento de resolución del problema.
\end{inparaenum} Una vez llevada a cabo esta tarea, pretendemos evaluar la
viabilidad de implementar una técnica de aprendizaje basada en análisis de
cláusulas de conflicto acorde a las necesidades de ambientes paralelos y
distribuidos, y su comparación empírica.

\bigskip

\noindent \textbf{Palabras claves:} Software engineering, Verificación and
validación, Sat-solvers, Parallelo y distribuido, Aprendizaje basado en
conflictos.


\cleardoublepage
%!TEX root = tesis.tex

%\begin{center}
%\large \bf \runtitle
%\end{center}
%\vspace{1cm}
\chapter*{\runtitle}

Software is nowadays an integral part of our lives. It participates in all
kind of tasks. From the sending of a text message or the realisation of a
phone call, to the control system of a nuclear plant or the cruise control
system of a car use systems based on hardware and software.

Software analysis is a field in software engineering that in general has the
responsibility of the application of formal methods in order to guaranty, or
at least gain some confidence in the absence of errors in a software artifact.
There exists a plethora of tools dedicated to that task, each one of them with
their particular biases for specific languages and methods to be applied.
Among them, there are many that have turn to the use of \ssolvers \emph{off-
the-shelf}  as a low level mechanism allowing the efficient exploration of a
finite space of models.
 
The aim of the present work is to implement a parallel and distributed
\ssolver that: \begin{inparaenum}[a)]   \item provides an adequate processing
workload balance at run-time over an \emph{a priori} unknown quantity and
configuration of equipment, \item provides an adequate balance of memory use,
both primary and secondary, at run-time over an unknown quantity and
configuration of equipment, \item is extensible and easy to parameterize in
all the relevant aspects that determines the behaviour of the tool in the
solution of the problem. \end{inparaenum} Once this task is done, we will
evaluate the viability of the implementation of learning techniques based in
conflict clause analysis according to the needs of parallel and distributed
scenarios, and their empirical comparison.

\bigskip

\noindenttextbf{Keywords:} Software engineering, Verification and validation,
\Sat-solvers, Parallel and distributed, Conflict based learning.


\cleardoublepage
%!TEX root = tesis.tex

\chapter*{Agradecimientos}

Intento determinar cuál fue el punto en el tiempo en el que todo empezó. No lo tengo muy claro. Mis recuerdos más viejos son para Nacho Regueira y Alejandro Abramoff. Cada uno con sus respectivas \texttt{XT} en las que hacíamos carteles con el Banner o jugábamos al Arkanoid ¿O tal vez fue en lo de Martín González? Un cumpleaños, sí. Una Commodore con un juego ¿Galaga tal vez? Después de eso, laguna, nada. Se me viene la 286, la primera compu ``mía''. Una flamante 286 con monitor color. Creo que fue un regalo de cumpleaños. Un regalo de mis viejos que era más bien para toda la familia. No recuerdo el día que llegó, pero sí recuerdo el día en que ``accidentalmente'' la \emph{formatié}. El técnico que vino aquella vez fue el primero que me enseñó algo de \emph{computación}. En aquella aprendimos Quattro Pro y Lotus123 con mi mamá. Después se me viene el lugar de los jueguitos, en el que parábamos con mi papá a veces cuando íbamos o veníamos de lo de la abuela Pocha. Creo que con el de Los Simpson nos agarramos el primer virus. El famoso Michellangelo. Me acuerdo también de cuando intenté, infructuosamente, instalarle Windows 95 a la pobre 286. Un Windows que me copié, diskette por diskette (creo que eran 16) de lo de Nadav Rajzman. Claramente nunca funcionó. En ese entonces ni me imaginaba que era imposible que funcione.

Lo próximo que recuerdo son los primeros años de la secundaria. Ya estaba cansado de la 286, que para ese entonces era más que vieja, así que me iba caminando al colegio para ahorrar hasta el último centavo de la mensualidad que me daba mi vieja. Así logré comprarme la Pentium 233 (¡con MMX!). Y con esa pasé los años en el Acosta, donde conocí a mucha gente que poco tuvo que ver con mi afinidad por la computación, pero mucho tuvo que ver con convertirme en la persona que soy, con todo lo que eso implica. La banda de Bande (en especial Mavi, Chuli, Diego, Juan, Santiago, Fede. Y Hernán, amigo de toda la vida), con la que, entre cafés con leche con dos de azucar y cervezas de \$1,60 perdimos el tiempo de la forma más maravillosa posible, entre amigos. Pepe y Manolo que siempre nos saludaban a la voz de ``Doctor''. Y pensar que algún día eso iba a tener un significado...

Después ``-¿Qué vas a estudiar?''. La pregunta más temida. ``-Ingeniería en informática...'', ``-Ajá'' y eso fue todo lo que dijo mi viejo. Tiempo después me dijo ``¿Por qué no te fijás? Hay una carrera de computación, en la facultad de Exactas, creo que se llama Computador científico...'' supongo que un arquitecto no podía permitir que su hijo fuera ingeniero. Para reforzar, en el CBC lo conocí a Damián, que estaba anotado para esa carrera de exactas en la que ``cuando entrás, lo primero que te dicen es que lo del CBC no sirve para nada, y estudiás todo de nuevo, pero bien'' ¡A mi juego me llamaron! Así llegué un día de Marzo de 2003, por primera vez, a Ciudad Universitaria. Todavía con la simultaneidad a cuestas. Y no me fui más.

Muchas cosas y mucha gente. Imposible hacerle honor a todas y todos. Aparece Damián nuevamente, consiguiéndome mi primer trabajo en el área. Él y Gutes (``el de ORT, que viene en auto''). Mi primer grupo de TP. El BebeJugando. Qué de chorizos a la pomarola, ñoquis caseros y asaditos que nos mandamos esos fines de semana de estudio entre Flores y Floresta (sin chiste por favor). Y cuando no nos juntábamos, mate en la mesa con mis hermanas, Tute, Chancho y Chuletas. Después aparecen Mati, Román, Piter, Tommy, Guido. Estábamos cursando (si se puede decir) AlgoII y empezamos a juntarnos a discutir ``cosas'' de la facultad junto con Charly y Nico Maur con quienes después fundamos la AEI. La agrupación con la que conquistamos el CoDep de computación por vaaaaarios años consecutivos. Mati y Piter de nuevo, AlgoIII. Gran cursada. Hace su aparición el piquetero programador, el remise con el TP, el té con tortas, Marmol y las pastas ¿Se acuerdan cuando nos ofendimos porque nos aprobaron un TP que claramente NO estaba bien?

Después la carrera se me va a un segundo plano. Sobresale la militancia. Itai y Fede. Charly y Nico de nuevo. Leo. Con quienes construimos una locura. A contramano y sin permiso. Con la bandera del pelado, un ``¿Por qué nacimos?¿Por qué luchamos?'' diseñado entre gallos y medianoches, a conquistar los sueños de una juventud inconforme, crítica, rebelde. Qué distante y a la vez, qué fundamental.

Y en el medio de la vorágine militante, se me aparece Ali ``esa, la que habló en la asamblea, la troska''. Resultó que troska no era. Pero sí fue (y es) mi compañera. Cuántas tardes estudiando en lo de Graciela. Bah, ella estudiaba y mientras me decía a mí ``dale, estudiá''. Cuántas noches en vela por el centro de estudiantes. Cuántos cronogramas, ajustes de cronogramas y ajustes de los ajustes de cronogramas que me armaste. Cuánta compañía, cuánta banca, cuánta paciencia, cuánto empuje. Y cuántas otras cosas para las que sobran las palabras.

Me estoy olvidando de muchos. Mis compañeros docentes. Estoy convencido de que aprendí más siendo docente que siendo estudiante. Se me vienen los que me ayudaron en mis primeros pasos: Schapa, Charly (ya dejá de aparecer che!), DFS. Aquellos con los que nos cargamos materias al hombro, Mati, Piter, Nico Rosner. La banda que hizo patria en Sistemas, Pablete, Mariano.

Los pibes del FEM!, en especial Marce con quien siempre compartimos interminables discusiones sobre la ciencia, la política científica y la historia universitaria. Manu y Nico, Lau, Lipa y Nano, Mauro, la primera camada. La CoNEAU ¿se acuerdan de la CoNEAU?

Los fieles de la bondiola. Dani V., Fran, Facu, Mica, Fede, Marianito y tantos otros de los que me estoy olvidando. Interminables comidas bajo la incumplible consigna de ``hoy no hablamos del DC''.

Toda la gente que ocupa, física o espiritualmente, la oficina 8. Mariano, Marcelo, Manu, Cani, Pau, Nico, Charly ¡Limpien el mate cuando se van!

Ya vengo llegando al final. Mis últimos recuerdos son para mis directores, Nico y Charly. Qué veranito ¡eh!

Al final, no sé en cuál de todos estos recuerdos está el punto de inflexión que me llevó hasta acá. Creo que fue un poco de todo. Lo que es seguro es que si uno de ustedes hubiera faltado, yo no habría sido el mismo, mi trabajo no habría sido el mismo. Por eso, a todos ustedes, muchas gracias. Más que muchas, infinitas. % OPCIONAL: comentar si no se quiere

\cleardoublepage
%!TEX root = tesis.tex
\hfill \textit{A mi viejo.}  % OPCIONAL: comentar si no se quiere

\cleardoublepage
\tableofcontents

\mainmatter
\pagestyle{headings}

\newcommand{\true}{\texttt{TRUE}\xspace}
\newcommand{\false}{\texttt{FALSE}\xspace}
\newcommand{\sat}{\emph{sat}\xspace}
\newcommand{\unsat}{\emph{unsat}\xspace}
\newcommand{\cnf}{\texttt{\textbf{CNF}}\xspace}
\newcommand{\npc}{\textbf{NP-Complete}\xspace}
\newcommand{\bt}{\emph{backtracking}\xspace}

\newcommand{\soft}{\emph{software}\xspace}
\newcommand{\hard}{\emph{hardware}\xspace}

\listoftodos

\chapter{Introducción}

En la actualidad el \soft forma parte integral de nuestra vida cotidiana. El
mismo interviene en todo tipo de teras. Desde el envío de un mensaje de texto o
la realización de una llamada telefónica hasta el sistema de control de una
central nuclear, pasando por el control de un ascensor o el sistema de velocidad
crucero de un automóvil, utilizan sistemas basados en \hard y \soft
informático. La variedad, alcance y \textbf{criticidad} de las responsabilidades
asignadas a las piezas de \soft hacen de las mismas un componente de alto
impacto en nuestra realidad actual. 

Por otra parte, la alta complejidad del \soft hace que la construcción del mismo
sea una tarea propensa a errores. El impacto de una falla en un componente de
\soft puede variar desde la imposibilidad de utilizar un artefacto doméstico
hasta una catástofre de magnitudes como la fundición del núcleo de un reactor
nuclear. 

Son estos aspectos los que motivan la necesidad creciente de construir métodos
(formalismos, metodologías, herramientas, etc.) que permitan garantizar la
calidad del \soft en un sentido general. La posibilidad de establecer la
ausencia de errores en un programa radica en la capacidad de establecer
inequívocamente que ese programa cumple una determinada propiedad. Por ejemplo
¿Será cierto que no es posible encender el microondas si la puerta está abierta?


\chapter{Preliminares}

\newcommand{\disj}[1]{\ensuremath{[#1]}}
\newcommand{\conj}[1]{\ensuremath{\{#1\}}}

\section{El problema de la SATisfactiblidad}

El problema de la satisfactibilidad (\emph{SAT problem}) consiste en determinar
si existe alguna asignación de valores de verdad (\true|\false) a cada una de
las variables booleanas que aparecen en una fórmula $\phi$ de la lógica
proposicional de modo que $\phi$ se haga verdadera. En caso de que dicha
asignación exista decimos que $\phi$ es satisfacible (o simplemente \sat) y la
asignación o valuación $\{ v_1 \leftarrow V_1, \ldots, v_n \leftarrow V_n \}$
correspondiente es un \textbf{modelo} de $\phi$. Es importante notar que en caso
de que una fórmula sea \sat podría existir más de una valuación que haga que la
fórmula sea verdadera. Si no existe ninguna valuación que haga verdadera a
$\phi$ decimos que $\phi$ es insatisfacible (o \unsat).

El problema de determinar si una fórmula $\phi$ dada es \sat o \unsat pertenece
a la clase de problemas \npc\cite{Cook:1971:CTP:800157.805047}. Es decir que no
se conoce un algoritmo cuya complejidad temporal sea polinomial que pueda determinar si una fórmula es \sat
o \unsat. La importancia que reviste el problema SAT para diversas áreas de las
ciencias de la computación, como ser el diseño de circuitos o la verificación
automática, ha impulsado el desarrollo de algoritmos relativamente eficientes
para una clase amplia de problemas conocidos como \emph{SAT solvers}.

\subsection{\emph{SAT Solving}}

Se conoce con el nombre de \emph{SAT solvers} a los algoritmos que permiten
determinar si una fórumla $\phi$ de la lógica proposicional tiene alguna
valuación que la haga verdadera. El algoritmo general de  \emph{SAT solving}
consiste en un algoritmo de \bt que podríamos esquematizar como sigue:

\missingfigure{Algoritmo DPLL}
% \begin{lstlisting}[mathescape,language=python]
% algoritmo?
% \end{lstlisting}

La mayoría de los \emph{SAT solvers} son
variaciones del algoritmo general
DPLL\cite{Davis:1962:MPT:368273.368557} y actúan sobre fórmulas
proposicionales expresadas en Forma Normal Conjuntiva (\cnf).

\subsection{Forma Normal Conjuntiva}

Decimos que una fórmula de la lógica proposicional se encuentra en Forma Normal
Conjuntiva (\cnf por sus siglas en inglés) si la misma es una conjunción de
disyunciones de literales; donde un literal es una variable $v$ o su negación
$-v$.

Por ejemplo, la fórmula $\phi = (p \vee q) \wedge (-p \vee q)$ se encuentra
en \cnf. Dado que los conectores lógicos están implícitos cuando una fórmula se
encuentra en \cnf normalmente escribimos $\phi = \conj{\disj{p, q}, \disj{-p,
q}}$ de modo equivalente.

\subsection{Algoritmo DPLL}

El algoritmo DPLL es un procedimiento 

\subsection{Aprendizaje}
\subsubsection{First UIP}
\subsubsection{Recorte de la base de datos}
\subsubsection{Actividad y LBD}

\section{Alloy}

\section{ParAlloy}
\subsection{BEDs}

\section{\emph{SAT Solving} distribuido}

\subsection{Aprendizaje}

\chapter{Técnica}



\chapter{Experimental}

\chapter{Conclusiones y trabajo futuro}

\section{Trabajo futuro}
- Portfolio

\bibliography{tesisbib}{}
\bibliographystyle{plain}

\end{document}
