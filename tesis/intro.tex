%!TEX root = tesis.tex

En la actualidad el \soft forma parte integral de nuestra vida cotidiana. El
mismo interviene en todo tipo de tareas. Desde el envío de un mensaje de texto o
la realización de una llamada telefónica hasta el sistema de control de una
central nuclear, pasando por el control de un ascensor o el sistema de velocidad
crucero de un automóvil, utilizan sistemas basados en \hard y \soft informático.
La variedad, alcance y \textbf{criticidad} de las responsabilidades asignadas a
las piezas de \soft hacen de las mismas un componente de alto impacto en nuestra
realidad actual.

Por otra parte, la alta complejidad del \soft hace que la construcción del mismo
sea una tarea propensa a errores. El impacto de una falla en un componente de
\soft puede variar desde la imposibilidad de utilizar un artefacto doméstico
hasta una catástofre de magnitudes como la fundición del núcleo de un reactor
nuclear.

Son estos aspectos los que motivan la necesidad creciente de construir métodos
(formalismos, metodologías, herramientas, etc.) que permitan garantizar la
calidad del \soft en un sentido general. La posibilidad de establecer la
ausencia de errores en un programa radica en la capacidad de establecer
inequívocamente que ese programa cumple una determinada propiedad que expresa el
adecuamiento del mismo al comportamiento esperado.
Por ejemplo si quisiéramos establecer el correcto  funcionamiento del mecanismo
de seguridad de un horno de microondas podríamos enunciar una propiedad como la
siguiente: ``No es posible encender el microondas si la puerta está abierta''.

La \textbf{verificación formal} de \soft consiste en demostrar que un
determinado sistema o algoritmo es correcto respecto a una
\textbf{especificación formal} del mismo. Los lenguajes naturales son
intrínsecamente ambiguos y por lo tanto una descripción de un sistema realizada
en lenguaje natural no puede ser verificada. Una especificación formal es una
descripción en un lenguaje formal de un sistema de \soft. Un lenguaje para ser
formal debe poseer: \begin{inparaenum}[a)] \item reglas para determinar cuándo
una sentencia pertenece al lenguaje (sintaxis) \item reglas que permitan
interpretar las sentencias (bien formadas) de manera precisa y significativa
(semántica) y \item reglas para inferir información útil a partir de la
especificación (\emph{proof theory})
\end{inparaenum}

El enfoque de la verificación formal de software, tal vez uno de los más
ambiciosos en el campo de la ingeniería de software, enfrenta dos problemas
fundamentales. En primer lugar está la dificultad de construir una
especificación formal que exprese de manera correcta el comportamiento esperado
del sistema (validación) ya que el mismo normalmente proviene de requerimientos
expresados en lenguaje natural. El segundo problema fundamental es que las
lógicas subyacentes a la mayoría de los lenguajes formales capaces de expresar
un conjunto lo suficientemente amplio e interesante de propiedades son
normalmente indecidibles \todo[inline]{¿Cita?}, es decir que no existe un
algoritmo capaz de, para toda fórmula del lenguaje, establecer si la misma es
verdadera o falsa.

El problema de la indecidibilidad se transforma en un escollo fundamental en
tanto impide la construcción de métodos de verificación formal completamente
automáticos. Diversos enfoques han sido desarrollados para atacar este problema
\todo[inline]{¿Citas?}. La mayoría \todo[inline]{¿Todos?} de estos métodos
consisten en restringir de alguna manera el lenguaje formal utilizado de modo de
generar un sublenguaje del mismo que sea decidible. Uno de los enfoques
desarrollados en este sentido es el conocido como \bmc
(verficación acotada de modelos).

\mc\cite{emerson:scp-2_3} es una técnica desarrollada para la verificación de
sistemas reactivos. La técnica consiste en la exploración exhaustiva y
automática del espacio de estados de un sistema modelado como una máquina de
estados finita. Cuando la cantidad de estados del sistema es relativamente
grande la exploración exhaustiva del espacio de estados se vuelve inpractible.
Esto motivó el desarrollo de la técnica conocida como \smc\cite{burch:lics90,
mcmillan93} que ataca el problema antedicho a partir de evitar construir el
grafo de estados del sistema utilizando fórmulas proposicionales para
caracterizar conjuntos y relaciones. En \smc el chequeo de las fórmulas
proposicionales es realizada típicamente mediante la manipulación de BDDs
(\bdds), una forma canónica de representar dichas fórmulas. A pesar de que las
técnicas de \smc basadas en  BDDs son capaces de manejar sistemas con cientos de
estados, el tamaño de los BDDs generados para sistemas grandes se vuelve
prohibitivo. 

El problema del espacio requerido por los BDDs motivó la creación de la técnica
conocida como \bmc\cite{Biere:1999:SMC:646483.691738} que reemplaza la
utilización de BDDs por la verificación de fórmulas proposicionales a través de
procedimientos SAT capaces de manejar en la actualidad fórmulas proposicionales
con varios miles de variables. En particular, la técnica presentada en
\cite{Biere:1999:SMC:646483.691738} consiste en buscar contraejemplos de una
longitud acotada $k$ a partir de generar una fórmula proposicional que sea
verdadera si y sólo si un contraejemplo de dicha longitud existe. Si bien esta
técnica es completa para sistemas modelados como máquinas de estados finitos ya
que el espacio de búsqueda, aunque exponencialmente grande, es finito, no es
cierto que esta metodología aplicada a un lenguaje formal general sea
completa\todo{Revisar esto}. La perdida de completitud en el caso general
generada por la introducción de la cota en la longitud del contraejemplo es lo
que, aplicado a un sistema lógico indecidible, permite obtener un subconjunto
del lenguaje original que ahora es decidible y, por lo tanto, automáticamente
verificable \todo{Revisar esto}. Un ejemplo típico de esto se da en lenguajes
donde la indecidibilidad se da a partir de la existencia de cuantificadores
universales sobre dominios no acotados. En estos casos la cota en la longitud
del contraejemplo se traduce en una cota para cada uno de los dominios no
acotados. De este modo las cuantificaciones universales se convierten en
cuantificaciones acotadas y por lo tanto el lenguaje se vuelve decidible (y en
particular codificable como una fórmula proposicional suceptible de ser
verificada automáticamente).

Debido a la incompletitud generada por la introducción de la cota en la longitud
del contraejemplo, \bmc no puede ser considerada una técnica de verificación en
el sentido estricto. Sin embargo la utilización de \bmc permite ganar confianza
en la propiedad que se quiere verificar antes de embarcarse en la costosa tarea
de demostrarla formalmente. Además \bmc ha sido aplicada con éxito a otras áreas
relacionadas como la generación automática de casos de test, etc \todo{Agregar
citas y sacar el etc.}.


\section{Alloy}

Existen una variedad de herramientas formales que utilizan el enfoque de \bmc
aplicado a la especificación y verificación de \soft. Alloy
\cite{jackson:acmtosem-11_2} es un lenguaje formal diseñado para expresar las
propiedades estructurales de un sistema. Posee una sintaxis declarativa lo
suficientemente poderosa como para expresar propiedades complejas y a la vez
plausible de ser analizada de forma completamente automática. De hecho, Alloy es
también una herramienta capaz de analizar automáticamente especificaciones
escritas en lenguaje Alloy.

\missingfigure{Ejemplo especificación Alloy}

\todo{Sarabastaza de Alloy hasta llegar a que se traduce a un problema SAT. Tal
vez mencionar extensiones como DynAlloy y casos de estudio exitosos}

\section{SAT Solving}

El interés en construir procedimientos sistemáticos y automáticos de
demostración se remonta al interés que tuvieron diversos matemáticos en
construir un método de esas características para la lógica de primer orden. Uno
de los máximos exponentes fue David Hilbert, quien enunció que la búsqueda de un
algoritmo de decisión para la lógica de primer orden era el problema central de
la lógica matemática. La demostración de que un procedimiento de estas
características no era posible elaborada por Church y Turing provocó una pérdida
de interés en esta clase de problemas. Sin embargo la construcción de
procedimientos, no ya de decisión, sino de demostración para la lógica de primer
orden (entre los cuales se destaca el desarrollado por Martin Davis y Hilary
Putnam \cite{Davis:1960:CPQ:321033.321034}) sumada a los crecientes campos de
aplicación dentro de las ciencias de la computación de lógicas decidibles o de
fragmentos decidibles de lógicas no decidibles, dieron un nuevo impulso al
desarrollo de estos métodos. 

En particular la variedad de aplicaciones que fueron surgiendo para la lógica
proposicional (\emph{Planning} en inteligencia artificial, demostración
automática de teoremas, \textbf{\mc}, \textbf{verificación de \soft}, etc.)
impulsaron fuertemente el desarrollo de técnicas y procedimientos para
determinar si una fórmula de la lógica proposicional podía ser verdadera. El
establecimiento de este problema como el primer problema \npc
\cite{Cook:1971:CTP:800157.805047} no impidió el desarrollo de heurísticas que
hoy permiten manejar fórmulas de cientos de miles de variables eficientemente.

Si bien existe una variedad de métodos para determinar si una fórmula
proposicional puede ser verdadera, los más utilizados son aquellos derivados del
algoritmo desarrollado por Davis, Longemann y Loveland (\dpll) en
\cite{Davis:1962:MPT:368273.368557} que es un refinamiento del algoritmo
presentado en \cite{Davis:1960:CPQ:321033.321034}. Este procedimiento es, en
esencia, un algoritmo de \bt.

Tuvieron que pasar varios años para que Marques y Silva introdujeran en
\cite{marques-silva:iccad96} la primera optimización radical al algoritmo \dpll
que dio origen a lo que hoy se conoce como procedimiento \cdcl (\CDCL) cuya base
es el aprendizaje de nuevas cláusulas ante cada conflicto. Ésto permite recortar el
espacio de búsqueda de manerea dramática. A partir de ese momento una serie de
optimizaciones han sido desarrolladas para cada una de las partes del algoritmo
desarrollado por Marques y Silva. La mayoría de estas optimización
\todo[inline]{¿Todas?} son heurísticas para cada uno de los puntos de decisión
del algoritmo \footnote{Para un relevamiento completo de las optimizaciones
aplicadas al algoritmo \CDCL ver \cite{manthey:mathesis}}.

A pesar de la mejora que representó la incorporación de aprendizaje propuesta
por Marques y Silva, con el correr del tiempo se estableció que los \ssolvers
invierten la mayor parte de su tiempo (cerca del $90\%$) en propagar las
implicaciones que se siguen de cada una de las decisiones tomadas durante la
búsqueda. Esta propagación incluye el barrido líneal \todo[inline]{Verificar
``lineal''} del conjunto de cláusulas que conforman el problema. Por lo tanto,
la incorporación de nuevas cláusulas (el \emph{leitmotiv} del procedimiento
\CDCL) a la vez que proporciona una ventaja al recortar el espacio de búsqueda
representa una desventaja en tanto que incrementa la cantidad de cláusulas sobre
las cuales es necesario realizar las propagaciones. Esta situación, sumado al
hecho de que la incorporación de nuevas cláusulas también representa un problema
desde el punto de vista de la memoria de la computadora, han llevado a la
comunidada a establecer que, por lo general, es mejor tener un política agresiva
de recorte de la base de datos de cláusulas aprendidas
\cite{Audemard:2009:PLC:1661445.1661509}. Si bien se han desarrollado distintas
heurísticas con muy buenos resultados, el problema de determinar cuántas y
cuáles cláusulas descartar (o su equivalente, preservar) sigue siendo un
problema ampliamente estudiado en el mundo \sat.

\section{SAT Solving paralelo y distribuido}

El establecimiento del problema \sat como un problema \npc hace pensar que no es
posible obtener un algoritmo polinomial que resuelva dicho problema. Además en
los últimos años la rapidez en el incremento de la velocidad de las computadoras
ha ido disminuyendo (esencialmente debido al problema de la disipación del calor
y el consumo de energía) y en su lugar se ha producido un incremento en la
cantidad de unidades de procesamiento disponibles así como su abaratamiento.
Estos hechos han impulsado a la comunidad \sat a explorar la posibilidad de
construir algortimos paralelos y dsitribuidos de \ssolving de modo de
\begin{inparaenum}[a)] \item sacar el máximo provecho de la nueva tendencia en
construcción de computadoras y \item proveer un algoritmo \textbf{escalable} que
permita aprovechar el abaratamiento del \hard \end{inparaenum}.

La aparición de los primeros \ssolvers paralelos y distribuidos
\cite{bohm:1996:afast, zhang:jsc-1996} es contemporánea a la aparición del
algoritmo \CDCL y por lo tanto estos algoritmos no atacaban el problema del
aprendizaje de nuevas cláusulas en este nuevo contexto.

$hola_{como^{te va}}$
