\documentclass{article}

\usepackage[spanish]{babel}
\usepackage[latin1]{inputenc}
\usepackage{framed}
\usepackage{paralist}
\usepackage[usenames]{color}
\usepackage{colortbl}

\usepackage[vcentering]{geometry}
\geometry{top=4cm, bottom=4cm, right=3cm, left=3cm}

\newcommand{\boxedcomment}[1]{%
\begin{framed}
 #1
\end{framed}
}

\begin{document}

\begin{Large}
 \begin{center}
    Fundamentos formales y herramientas para el desarrollo de software orientado a servicios
 \end{center}

\end{Large}


\section{Objetivos}
  El mundo del desarrollo de software se ha vuelto intr\'insecamente heterog\'eneo. Por un lado, los analistas y dise\~nadores tienen a su disposici\'on una mir\'iada de lenguajes y notaciones para capturar y modelar distintos aspectos del software. Por ejemplo, UML \cite{omg-sysml04,omg-ocl04} ofrece un conjunto de notaciones en forma de diagramas que van desde diagramas de clases hasta diagramas de estado, diagramas de colaboraci\'on, etc. Esta proliferaci\'on refleja la necesidad de reducir la complejidad del desarrollo de grandes sistemas ya que cada lenguaje permite a los ingenierios concentrarse en una vista o etapa espec\'ifica del proceso de desarrollo. Lo mismo ocurre en el nivel de los formalismos que proveen soporte formal al uso de dichos lenguajes y m\'etodos: se ha utilizado l\'ogica ecuacional para tipos de datos, l\'ogica temporal para comportamiento reactivo, funciones de alto orden para seguridad, etc. Como consecuencia de esto existen muchas notaciones y lenguajes de entrada distintos para 
las herramientas. A su vez, los usuarios tambi\'en tienen sus h\'abitos y preferencias por lenguajes particulares. En resumen, el escenario que enfrentamos hoy en d\'ia es el de una multitud de lenguajes de modelado y l\'ogicas de soporte, y de herramientas para el procesamiento de dichos lenguajes mediante el razonamiento en las l\'ogicas subyacentes.
  
  Estos entornos de dise\~no y desarrollo introducen a la heterogeneidad como la principal fuente de complejidad ya que la misma torna sumamente dif\'icil, casi imposible, comprender c\'omo los sistemas se comportar\'an e interactuar\'an con las personas y con otros sistemas. Esta es la raz\'on por la cual, en los \'ultimos a\~nos, el campo de los fundamentos de las ciencias de la computaci\'on ha presenciado un crecimiento del inter\'es en el problema de lograr proveer fundamentos formales para el dise\~no de software heterog\'eneo.
  
  Por otro lado, en el actual contexto de computaci\'on global ubicua, la estructura de los sistemas de software se est\'a volviendo m\'as y m\'as din\'amica ya que las aplicaciones necesitan poder responder y adaptarse a los cambios en el entorno en el que operan. Por ejemplo el nuevo paradigma de software orientado a servicios (SOC) soporta una nueva generaci\'on de aplicaciones de software que se ejecutan  sobre infraestructuras de red y computacionales disponibles globalmente desde las cuales pueden obtener servicios din\'amicamente (sujeto a una negociaci\'on de un \emph{Service Level Agreement} - SLA) y unirse a los mismos de modo que, colectivamente, puedan alcanzar los objetivos de negocio dados. No existe nin\'un control sobre la naturaleza de los componentes a los que una aplicaci\'on puede unirse. En particular, el desarrollo ya no se lleva a cabo de una manera \emph{top-down} en la que los subsistemas son desarrollados e integrados por ingenieros preparados: en SOC, el descubrimiento y la uni\'on 
son realizadas por el \emph{middleware}. Por lo tanto, no es necesariamente cierto que este nivel extra de heterogeneidad, derivado de cosas que ocurren en tiempo de ejecuci\'on, pueda ser manejado a trav\'es de traducciones (en tiempo de dise\~no) a un lenguaje com\'un. 
  
  Desde un punto de vista general, los objetivos del proyecto se centran en hacer contribuciones en dos campos a partir de:
  \begin{itemize}
   \item proveer fundamentos formales para el dise\~no y desarrollo de software orientado a servicios soportando esta nueva concepci\'on de heterogeneidad derivada del mismo, y
   \item el desarrollo de herramientas para el dise\~no, valiaci\'on y verificaci\'on de este tipo de software
  \end{itemize}

  Considerando que una parte significativa del software que se desarrolla en la actualidad encaja en este nuevo paradigma, creemos que la investigaci\'on en este campo tendr\'a un importante impacto en la b\'usqueda de acortar la distancia entre los m\'etodos formales y el desarrollo industrial de software.
  
  El proyecto propone entonces el desarrollo de los dos aspectos principales se\~nalados anteriormente (la provisi\'on de fundamentos formales capaces de lidiar con la heterogeneidad derivada de la variedad de lenguajes con que el software es dise\~nado, y c\'omo el mismo se integra en tiempo de ejecuci\'on y la provisi\'on de herramientas que soporten el an\'alisis de dichas descripciones bas\'andose en dichos fundamentos) en una forma interrelacionada. En cuanto al problema de lidiar con las descripciones heterog\'eneas de las piezas de software proponemos el desarrollo de un \emph{framework} te\'orico m\'as flexible que el basado en la representaci\'on con instituciones. El objetivo de esto es subsanar las limitaciones provenientes del hecho de que dicha representaci\'on est\'a enfocada en resolver la integraci\'on en tiempo de dise\~no: una instituci\'on es la formalizaci\'on de una l\'ogica, una representaci\'on mediante instituciones provee los medios para traducir una l\'ogica en otra de modo que se 
preserve la sem\'antica. Normalmente, la heterogeneidad es tratada mediante la traducci\'on de diferentes lenguajes a un \'unico lenguaje m\'as expresivo en la que las diferentes vistas son homogeneizadas, es decir, integradas en una \'unica descripci\'on. El nuevo \emph{framework} se basar\'a en la noci\'on de interoperabilidad de los lenguajes y atacar\'a tres importantes limitaciones de los enfoques actualmente disponibles:
  \begin{inparaenum}[\itshape a\upshape)]
  \item el hecho de que, debido a que las instituciones fueron definidas para dar cuenta de las l\'ogicas, las mismas no son, necesariamente, la abstracci\'on adecuada para manejar los lenguajes y notaciones utilizados para el modelado o especificaci\'on de sistemas;
  \item la falta de una noci\'on apropiada de especificaci\'in para sistemas heterog\'eneos que soporte \emph{dynamic (service-oriented) binding}; y
  \item la carencia de herramientas que soporten adecuadamente los m\'etodos formales basados en este nuevo concepto
  \end{inparaenum}
  
  El desarrollo de un \emph{framework} con estas caracter\'isticas requerir\'a de una serie de tareas. El primer objetivo es el estudio del estado del arte en el campo de las especificaciones heterog\'eneas de modo de identificar sus limitaciones para lidiar con lenguages no l\'ogicos como UML o SCA. La literatura existente sobre especificaciones heterog\'eneas se concentra en lenguajes l\'ogicos formalizados como instituciones \cite{goguen:cmwlp84} y en traducciones con preservaci\'on de la sem\'antica entre ellos, formalizados como morfismos entre instituciones \cite{goguen:jacm-39_1} y representaciones entre instituciones \cite{tarlecki:sadt-rtdts95}. Las instituciones de Grothendieck \cite{diaconescu:acs-10_4} son un paso adelante en el campo de las especificaciones heterog\'eneas porque internalizan las traducciones como mrofismos heterog\'eneos entre teor\'ias introduciendo el concepto de lenguaje heterog\'eneo cuyo modelo ser\'a interpretado de acuerdo con las proyecciones inducidas por estos 
morfismos en la correspondiente clase de modelos.
  
  REVISAR -
  En muchos casos uno puede idear codificaciones de dichos lenguajes en instituciones al costo de perder el significado de una gram\'atica o teor\'ia de modelos. Por ejemplo, las instituciones fuerzan que los mapeos sean transformaciones naturales sobre functores gram\'aticos, lo que es demasiado restrictivo en situaciones donde es posible encontrar traducciones (usualmente parciales) entre REVISAR - oraciones - REVISAR (no estructuradas). En el nivel de las teor\'ias de modelos, la condici\'on de satisfactibilidad (una propiedad estructural que requiere que la satisfactibilidad de los predicados sea independiente de las firmas) impide la definici\'on de mapeos que operan sobre subdominios que son dependientes de las firmas.
  - REVISAR
  
  Es por esto que proponemos trabajar hacia una noci\'on de formalismos de especificaci\'on m\'as flexible (como la de categor\'ias coordenadas de Fiadeiro o las l\'ogicas de especificaci\'on de Ehrig) y desarrollar una noci\'on de interoperabilidad  que (BLA BLA BLA) sino a una noci\'on de conector similar a la usada en arquitectura de software. Un conector (un concepto conocido en arquitectura del software) debe soportar un cierto nivel de interoperabilidad a partir de declarar en sus roles los elementos que son necesarios abstraer de los lenguajes que ser\'an  conectados (que pueden ser tanto sint\'acticos como sem\'anticos, o una mezcla de ambos) y el mecanismo a trav\'es del cual se lleva a cavo la integraci\'on.

  Uno de los objetivos del proyecto es desarrollar bases te\'oricas similares a aquellas existentes para los lenguajes l\'ogicos pero para lenguajes no l\'ogicos bas\'andose en el concepto de interoperabilidad en lugar de en el de representaci\'on de lenguajes.
  
  Tambi\'en pretendemos incluir otras caracter\'isticas que han sido extensamente estudiadas para los lenguajes l\'ogicos, como mecanismos de estructuraci\'on de especificaciones \cite{borzyszkowski:tcs-286_2,lopezpombo:ictac10}. Los mecanismos de estructuraci\'on proveen formas de copmrender a las teor\'ias como resultado de un proceso din\'amico de dise\~no de software. Creemos que esta es la forma correcta de comprender c\'omo se construyen los sistemas, de un modo incremental y evolutivo. Por lo tanto esto los convierte en una necesidad  a la hora de impulsar esta concepci\'on novedosa sobre el dise\~no de software heterog\'eneo, especialmente si queremos que esta metodolog\'ia sea adoptada en la pr\'actica del dise\~no y desarrollo de software.
  
  Sobre los mecanismos de estructuraci\'on, se debe tener en cuenta que los mismos deben operar tambi\'en en canales. En particular, proponemos revisar la noci\'on de conector de alto orden desarrollada en \cite{lopez+:acmtosem-12_1} para arquitecturas de software que est\'a inspirada en la noci\'on de m\'odulo de alto orden de las especificaciones algebraicas.
  
  El proyecto entonces estar\'a centrado en el desarrollo de los fundamentos te\'oricos para la construcci\'on y composici\'on de especificaciones heterog\'eneas bas\'andose en el concepto de interoperabilidad (en lugar de en la representaci\'on de lenguajes) que puede ser usado tanto en tiempo de disen\~no como de ejecuci\'on, y en la validaci\'on de esos fundamentos sobre aquellas notaciones de la ingenir\'ia del software, m\'etodos y lenguajes en los que la heterogeneidad es un aspecto clave. Para lograr esto proponemos:

  \begin{itemize}
  \item establecer las limitaciones del enfoque tradicional basado en instituciones tanto para actuar como \emph{framework} formal para lenguages no l\'ogicos como para lidiar con conceptos como la interoperabilidad o el \emph{dynamic binding}, ambos esenciales en las arquitecturas orientadas a servicios,
  
  \item definir los fundamentos formales para un \emph{framework} heterog\'eneo basado en el concepto de interoperabilidad. Esto es particularmente desafiante y novedoso en tanto que, contrariamente a lo posibilitado por el enfoque basado en instituciones, la integraci\'on en tiempo de ejecuci\'on no puede ser resuelta de forma est\'atica durante la etapa de dise\~no,
  
  \item desarrollar el prototipo de una herramienta que soporte el dise\~no de software heterog\'eneo orientado a servicios.
  \end{itemize}

\section{Antecedentes}

\section{Actividades y metodolog\'ia}

\section{Factibilidad}

\section{Bibliograf\'ia}


\bibliographystyle{alpha}
\bibliography{bibdatabase}

\end{document}