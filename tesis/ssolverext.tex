%!TEX root = tesis.tex

\subsection{La estrategia implementada}
% - Noción de problema difícil vs problema fácil
% - Imposibilidad de determinar o siquiera estimar eso a priori
% - Enorme varianza entre los subproblemas de un problema

% - Cuándo declarar que un problema es "demasiado difícil" (para
% atacarlo sin partirlo)?
% - Cuáles de los solvings en curso abortar? Cuántos? Cuándo?

%         - Imposibilidad de determinar o siquiera estimar a priori (en base a
%                 criterios estáticos) la dificultad de un (sub)problema.

%         - No sabemos cuánto va a tardar.
%         - No sabemos si es fácil o difícil.

%         - Partir temprano algo fácil (una rama que con poco más se cerraba)
%           es un error que se paga caro, sobre todo si se repite recursivamente.

%         - Invertir mucho esfuerzo en algo difícil y encima no lograr cerrarlo
%           (sólo para terminar partiéndolo igual, y mucho después) es otro
%           error que se paga caro, sobre todo si se repite recursivamente.

%         - Invertir mucho esfuerzo en algo difícil hasta cerrarlo sin partirlo
%           no siempre es una buena idea (atenta contra el paralelismo,
%           sube el camino crítico, etc).

%         - No existe un valor de TO prefijado que sea adecuado para cualquier
%           problema raíz (y no queremos que el usuario tenga que proveer uno).

%         - Incluso para cierto problema raíz R, no existe un único valor de TO
%           que sea adecuado durante todo el transcurso de la corrida.

%                 - Cada vez que se parte un subproblema (sup. elección razonable de
%                   vars) se obtienen hijos más fáciles, pero

%                     - con mucha varianza en su distribución [tablita pamela8?]

%                     - con tan poca predecibilidad como dijimos antes respecto de
%                       la tasa (hijo_mas_caro/padre)

%                 - O sea que lo único que realmente sabemos es que "se van haciendo
%                   más fáciles" (pero no a qué velocidad, ni cuánto más rápido a lo
%                   largo de una rama vs. de otra, etc).

%         => Es necesario algún criterio dinámico / adaptativo / etc.

%         - Podría basarse en
%                 - aprendizaje sobre el problema (qué pasó hasta ahora otras veces, etc)
%                 - feedback loop observando métricas del comportamiento/estado del sistema


% - Cómo partir un problema demasiado difícil?

%         - Cuáles y cuántas variables levantar

%                 => Gran Temón en sat-solving, recontra determinante para la
%                   cant/dificultad/distribución de los subproblemas, etc
%                   VSIDS, etc.

%                 => Nivel de agresividad, fanout, potencial explosión;
%                         trade-off "muchos" vs "más fáciles" (ojalá!);
%                         hay que tener cuidado con el fanout recursivo.

%         - Filtrado de subproblemas triviales

%                 - Rara vez se da el peor caso 2^n totalmente denso
%                 - Muchas veces se puede lograr mucho menos!
%                 - Hay maneras muy baratas de filtrar los muy triviales
%                 - Hay maneras (no tan) baratas de filtrar los (no tan) triviales
%                 => Trade-off:
%                         - escaso filtrado =>
%                             overhead innecesario por proliferación de tareas fáciles evitables
%                         - exceso de esfuerzo en filtrado =>
%                             excesiva centralización de costo computacional que podría distribuirse


% - Ni bien alguien queda ocioso debe asignársele más trabajo.

% - Decisión importante: ¿cuál de todas las tareas pendientes es la "próxima"?
%   => afecta cómo se recorre el espacio de búsqueda
%   => termina afectando el scheduling (en presencia de feedback loops etc)

% - Además, minimizar costos de movimientos de tareas innecesarios
%   => si ese alguien ya tiene trabajo pendiente local, podría ser bueno
%      que se le asigne lo mejor posible dentro de lo local

Comenzaremos esta sección con un desglosando las principales cuestiones a
resolver por una estrategia completamente automática para nuestra herramienta.
Surge así la primer pregunta relevante que cualquier estrategia debe
responder: ¿Cuándo declarar que un (sub)problema es demasiado difícil para
atacarlo como tal? Es decir ¿Cuándo se toma la decisión de partir un problema
en nuevos subproblemas?

\subsubsection{Dificultad de un problema}

En este punto es importante recalcar el hecho de que la partición de un
problema en nuevos subproblemas se lleva a cabo con el objetivo de distribuir
las distintas porciones del espacio de búsqueda entre distintas unidades de
cómputo. Así, invertir demasiado tiempo en intentar obtener un resultado
secuencialmente para un problema difícil atenta prohibitivamente contra el
paralelismo. Por lo tanto la decisión de cuándo abortar un \solving en curso
se vuelve crucial.

Si un problema va a tomar demasiado tiempo no tiene sentido invertir tiempo de
\solving secuencial en intentar resolverlo y es conveniente partirlo
\emph{cuanto antes}. Por otra parte, partir un problema en un momento tal que
si hubiéramos esperado una fracción  pequeña de tiempo más el \w habría
arribado a un resultado para el problema implica que \emph{todo} el tiempo de
cómputo previamente invertido es desperdiciado. Estos dos errores se vuelven
catastróficos cuando se repiten recursivamente\todo{¿Se entiende
``recursivamente''?}.

Subyace a toda esta cuestión la pregunta fundamental de cómo determinar que un
problema dado es fácil o difícil. Si tuviéramos algún criterio que nos
permitiese siquiera aproximar la dificultad de un problema dado, o el
porcentaje de avance durante una corrida secuencial de un \ssolver, podríamos
elaborar una estrategia que atienda razonablemente a las inquietudes
planteadas arriba. Lamentablemente no se conoce ninguna métrica que permita
establecer \apriori la dificultad de un problema. Si bien la complejidad de
los algoritmos de \ssolving se expresa en función de la cantidad de variables
de un problema (y es exponencial), la experiencia indica que existen problemas
con pocas variables sumamente difíciles y problemas con muchas variables
razonablemente sencillos.

Resulta claro entonces que una primera aproximación \emph{naive} a este
problema como podría ser la elección de un \tout fijo como criterio para
determinar cuándo un problema debe ser partido, resulta una mala alternativa.
La disparidad en la dificultad de los distintos problemas hace que sea
imposible atacar razonablemente todos los problemas con un único valor de \tout.
Una posibilidad sería entonces que el usuario seleccionara el \tout que quiere
utilizar para cada problema. Sin embargo esto atenta sustancialmente contra la
usabilidad de la herramienta ya que requiere asumir un \emph{expertise} por
parte del usuario y un conocimiento previo del problema a resolver que no son
razonables. 

Además a medida que partimos recursivamente un problema, los subproblemas
generados --si la elección de variables a levantar fue apropiada-- son
\emph{cada vez} más fáciles. Por lo tanto un valor de \tout que resulta
adecuado en un momento no lo será más adelante. Esto podría llevarnos a
elaborar un criterio un tanto más refinado que lo anterior. Por ejemplo
podríamos optar por tener un valor de \tout para cada \emph{nivel} del
problema. Sin embargo surge nuevamente el mismo problema que mencionábamos
anteriormente. Cada vez que un problema es partido, los subproblemas generados
difieren notablemente en su nivel de dificultad. Aún más, las distribuciones
de dificultad entre los hijos de dos subproblemas distintos de un mismo
problema (a misma cantidad de variables levantadas) pueden ser muy distintas.
Por lo tanto, este refinamiento tampoco resulta adecuado.

En general, la ausencia de un criterio estático para determinar la dificultad
de un problema dado, hace imposible la elección de un criterio de corte que
sea también estático. Por lo tanto se vuelve necesario que el criterio para
determinar que un problema debe ser partido sea dinámico. Es decir que el
mismo se vaya adaptando durante la ejecución. Un criterio de estas
características tiene dos fuentes de información fundamentales. Por un lado el
estado general del sistema y su evolución (su historia). Esto puede incluir
métricas sobra la carga del sistema, cantidad de problemas pendientes, espacio
de almacenamiento disponible, etc. La segunda fuente de información surge de
las características del problema que sean observables a lo largo de la
ejecución. Ejemplos de esto son la cantidad de subproblemas producidos cada
vez que se partió un problema, estadísticas sobre los tiempos que demoraron
los subproblemas ya resueltos, etc. 

Cualquier estrategia automática razonable deberá entoces dar respuesta a la
problemática de decidir cuándo partir un problema teniendo en cuenta las
observaciones mencionadas. Una vez resuelto este dilema surge la segunda
pregunta fundamental a responder que es ¿Cómo partir un problema? Que
esencialmente se traduce en ¿Cuántas y qué variables levantar en cada ocasión?

\subsubsection{Criterio de particionado}

\newcommand{\vsids}{VSIDS\xspace}

La determinación de qué variables levantar en el momento en que se decide
partir un problema es uno de los grandes temas en el mundo del \ssolving. En
un sentido, este problema es equivalente al problema que enfrentan los
\ssolver secuenciales cuando deben tomar una decisión. Cuando un \ssolver
secuencial agotó las propagaciones de la última decisión tomada, debe tomar
una nueva decisión. Tomar una decisión implica en primera instancia
seleccionar una variable y luego seleccionar cuál de los dos posibles valores
de verdad asignar a esa variable. En nuestro caso, dado que al levantar una
variable los dos valores de verdad son considerados \emph{simultáneamente}, el
problema se reduce a elegir cuidadosamente la variable a levantar. Uno de los
métodos más difundidos para elegir la próxima variable en los \ssolvers
secuenciales, se basa en un criterio de actividad conocido como \vsids. Del
mismo modo que para las cláusulas --ver \ref{sec:longactlbd}-- cada vez que
una variable \emph{participa} de un conflicto su actividad es incrementada.
Luego, cuando el \ssolver se encuentra en la situación de tener que elegir una
variable para continuar la búsqueda, elige aquella con mayor actividad.

La elección de las variables a ser levantadas es crucial. Como se ve en
\ref{tablita???} una buena elección de variables puede generar una
descendencia muy fácil o muy difícil. En general, no se posible saber cuál es
la mejor elección de variables a ser levantadas. Sin embargo la heurística
\vsids a reportado muy buenos resultados en el mundo del \ssolving secuencial.

Además de la determinación de qué variables levantar, es importante también
tener en cuenta cuántas levantar en cada momento. En primer lugar es
importante tener en cuenta que la cantidad potencial de subproblemas generados
es exponencial con respecto a las variables levantadas. Esto afecta en un
doble sentido. En primera instancia es necesario observar que, dado que el
proceso de particionado se repite recursivamente, la generación de una
excesiva cantidad de subproblemas ante cada particionado genera una explosión
exponencial de problemas. Si este factor no es tenido en cuenta es altamente
probable que el sistema diverja y no sea capaz de arribar a una solución.
Sumado a esto, el tiempo insumido en generar la progenie\todo{:)} de un
problema puede volverse excesivamente grande atentando una vez más contra el
paralelismo y la eficiencia de la herramienta.

Una vez más debemos recordar que la idea de levantar variables se basa en que,
ante una elección razonable de cuáles variables levantar, a mayor cantidad de
variables levantadas, más fáciles serán los subproblemas generados. Es por
esto que si la cantidad de variables a levantar es demasiado pequeña, es
probable que los subproblemas generados no sean suficientemente fáciles para
ser atacados y que, por lo tanto, requieran ser partidos ellos también en el
futuro haciendo que el tiempo invertido en intentar resolverlo se convierta en
tiempo desperdiciado. Además, si la cantidad de variables levantadas es muy
baja, existe la posibilidad de que la cantidad de subproblemas generados no
sea suficiente para aprovechar al máximo el \hard disponible.

La última observación referida al criterio de particionado tiene que ver con
qué problemas son dignos de ser distribuidos realmente. Hemos mencionado
muchas veces que al levantar $n$ variables, la cantidad \textbf{potencial} de
subproblemas generados es $2^n$. El uso de la palabra potencial tiene que ver
con el hecho de que varios de los subproblemas generados al levantar $n$
variables podrían ser resueltos trivialmente ¿A qué nos referimos con
trivialmente? En primer lugar a aquellos subproblemas en los que alcanza con
propagar\footnote{El uso del término propagación en este contexto es estricto.
Es decir que nos referimos exclusivamente a la realización de la clausura de
\bcp sobre el subproblema. Esta propagación no incluye la toma de nuevas
decisiones.} las implicaciones de las decisiones tomadas para obtener un
resultado --\sat o \unsat--. Pero más en general, nos referimos a todos
aquellos subproblemas en los que el \emph{overhead} de generar la tarea, más
trasladarla a otro \w, más cargar la tarea en el nuevo \w, más \solvear dicho
problema hasta arribar a un resultado sea suficientemente grande como para que
sea conveniente atacarlo directamente hasta arribar a un resultado sin generar
una nueva tarea.

Se introduce entonces una nueva variable en el criterio de particionado que es
cuáles de los potenciales $2^n$ subproblemas generar como tareas a ser
atacadas por otros \ws y cuáles atacar directamente sin generarlos como nuevas
tareas. Se debe observar que \emph{no generar} un subproblema requier invertir
un tiempo de cómputo suficiente para determinar el resultado de ese
subproblema. Una estrategia automática debe entonces decidir qué hacer con
esta variable teniendo en cuenta que invertir poco o nada de tiempo de cómputo
en filtrar los subproblemas puede degradar significativamente la eficiencia
del sistema debido al \emph{overhead} invertido en resolver problemas
demasiado fáciles. Pero también debe tener en cuenta que invertir demasiado
tiempo de cómputo en el filtrado de los subproblemas tiende a degradar
significativamente el paralelismo.

\begin{itemize}
	\item Si idle se parte el más viejo
	\item La próxima a resolver es:
		\begin{itemize}
			\item Si no tengo tareas locales:
				\begin{itemize}
					\item Bajo la tarea que corresponde por BFS +
					\item $\frac{\sharp tasks}{\sharp workers}$ del que más tiene (límite 10)
				\end{itemize}
			\item Si tengo: Agarro la que corresponde por BFS
		\end{itemize}
	\item Parto cuando la frecuencia de $\sharp UNSATs/_s$
	\item Detalles:
	\begin{itemize}
		\item Target de UNSATs por esgundo
		\item Frecuencia de checkeo
		\item Tamaño ventana
	\end{itemize}
\end{itemize}

\subsection{Decisiones que vale la pena seguir investigando}

\section{Resultados experimentales}

\begin{table}[h]\tiny
	\begin{tabular}{lrrrrrr}
		\toprule
		problem	&	scope	&	sequential runtime	&	parallel walltime	&	parallel overhead	&	speedup	&	efficiency \\
		\cmidrule(r){1-7}
		Pamela	&	8	&	308.26	&	60.46	&	3561.23	&	5.10x	&	0.08 \\
		Pamela	&	9	&	76168.16	&	407.34	&	-50098.63	&	186.99x	&	2.92 \\
		Pamela	&	10	&		&		&	0.00	&		&	 \\
		\cmidrule(r){1-7}
		Closure	&	11	&	749.65	&	291.28	&	17891.95	&	2.57x	&	0.04 \\
		Closure	&	12	&	3983.36	&	1914.45	&	118541.30	&	2.08x	&	0.03 \\
		Closure	&	13	&		&		&	0.00	&		&	 \\
		\cmidrule(r){1-7}
		MarkGC Soundness2	&	9	&	217.31	&	200.85	&	12637.31	&	1.08x	&	0.02 \\
		MarkGC Soundness2	&	10	&	2855.3	&	1376.89	&	85265.47	&	2.07x	&	0.03 \\
		\bottomrule
	\end{tabular}
	\caption{Tiempo de ejecución (en segundos) distribuido vs. secuencial}
	\todo[inline]{Resultados parciales. Completar con todos los experimentos}
\end{table}