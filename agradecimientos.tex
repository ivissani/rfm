%!TEX root = tesis.tex

\chapter*{Agradecimientos}

Intento determinar cuál fue el punto en el tiempo en el que todo empezó. No lo tengo muy claro. Mis recuerdos más viejos son para Nacho Regueira y Alejandro Abramoff. Cada uno con sus respectivas \texttt{XT} en las que hacíamos carteles con el Banner o jugábamos al Arkanoid ¿O tal vez fue en lo de Martín González? Un cumpleaños, sí. Una Commodore con un juego ¿Galaga tal vez? Después de eso, laguna, nada. Se me viene la 286, la primera compu ``mía''. Una flamante 286 con monitor color. Creo que fue un regalo de cumpleaños. Un regalo de mis viejos que era más bien para toda la familia. No recuerdo el día que llegó, pero sí recuerdo el día en que ``accidentalmente'' la \emph{formatié}. El técnico que vino aquella vez fue el primero que me enseñó algo de \emph{computación}. En aquella aprendimos Quattro Pro y Lotus123 con mi mamá. Después se me viene el lugar de los jueguitos, en el que parábamos con mi papá a veces cuando íbamos o veníamos de lo de la abuela Pocha. Creo que con el de Los Simpson nos agarramos el primer virus. El famoso Michellangelo. Me acuerdo también de cuando intenté, infructuosamente, instalarle Windows 95 a la pobre 286. Un Windows que me copié, diskette por diskette (creo que eran 16) de lo de Nadav Rajzman. Claramente nunca funcionó. En ese entonces ni me imaginaba que era imposible que funcione.

Lo próximo que recuerdo son los primeros años de la secundaria. Ya estaba cansado de la 286, que para ese entonces era más que vieja, así que me iba caminando al colegio para ahorrar hasta el último centavo de la mensualidad que me daba mi vieja. Así logré comprarme la Pentium 233 (¡con MMX!). Y con esa pasé los años en el Acosta, donde conocí a mucha gente que poco tuvo que ver con mi afinidad por la computación, pero mucho tuvo que ver con convertirme en la persona que soy, con todo lo que eso implica. La banda de Bande (en especial Mavi, Chuli, Diego, Juan, Santiago, Fede. Y Hernán, amigo de toda la vida), con la que, entre cafés con leche con dos de azucar y cervezas de \$1,60 perdimos el tiempo de la forma más maravillosa posible, entre amigos. Pepe y Manolo que siempre nos saludaban a la voz de ``Doctor''. Y pensar que algún día eso iba a tener un significado...

Después ``-¿Qué vas a estudiar?''. La pregunta más temida. ``-Ingeniería en informática...'', ``-Ajá'' y eso fue todo lo que dijo mi viejo. Tiempo después me dijo ``¿Por qué no te fijás? Hay una carrera de computación, en la facultad de Exactas, creo que se llama Computador científico...'' supongo que un arquitecto no podía permitir que su hijo fuera ingeniero. Para reforzar, en el CBC lo conocí a Damián, que estaba anotado para esa carrera de exactas en la que ``cuando entrás, lo primero que te dicen es que lo del CBC no sirve para nada, y estudiás todo de nuevo, pero bien'' ¡A mi juego me llamaron! Así llegué un día de Marzo de 2003, por primera vez, a Ciudad Universitaria. Todavía con la simultaneidad a cuestas. Y no me fui más.

Muchas cosas y mucha gente. Imposible hacerle honor a todas y todos. Aparece Damián nuevamente, consiguiéndome mi primer trabajo en el área. Él y Gutes (``el de ORT, que viene en auto''). Mi primer grupo de TP. El BebeJugando. Qué de chorizos a la pomarola, ñoquis caseros y asaditos que nos mandamos esos fines de semana de estudio entre Flores y Floresta (sin chiste por favor). Y cuando no nos juntábamos, mate en la mesa con mis hermanas, Tute, Chancho y Chuletas. Después aparecen Mati, Román, Piter, Tommy, Guido. Estábamos cursando (si se puede decir) AlgoII y empezamos a juntarnos a discutir ``cosas'' de la facultad junto con Charly y Nico Maur con quienes después fundamos la AEI. La agrupación con la que conquistamos el CoDep de computación por vaaaaarios años consecutivos. Mati y Piter de nuevo, AlgoIII. Gran cursada. Hace su aparición el piquetero programador, el remise con el TP, el té con tortas, Marmol y las pastas ¿Se acuerdan cuando nos ofendimos porque nos aprobaron un TP que claramente NO estaba bien?

Después la carrera se me va a un segundo plano. Sobresale la militancia. Itai y Fede. Charly y Nico de nuevo. Leo. Con quienes construimos una locura. A contramano y sin permiso. Con la bandera del pelado, un ``¿Por qué nacimos?¿Por qué luchamos?'' diseñado entre gallos y medianoches, a conquistar los sueños de una juventud inconforme, crítica, rebelde. Qué distante y a la vez, qué fundamental.

Y en el medio de la vorágine militante, se me aparece Ali ``esa, la que habló en la asamblea, la troska''. Resultó que troska no era. Pero sí fue (y es) mi compañera. Cuántas tardes estudiando en lo de Graciela. Bah, ella estudiaba y mientras me decía a mí ``dale, estudiá''. Cuántas noches en vela por el centro de estudiantes. Cuántos cronogramas, ajustes de cronogramas y ajustes de los ajustes de cronogramas que me armaste. Cuánta compañía, cuánta banca, cuánta paciencia, cuánto empuje. Y cuántas otras cosas para las que sobran las palabras.

Me estoy olvidando de muchos. Mis compañeros docentes. Estoy convencido de que aprendí más siendo docente que siendo estudiante. Se me vienen los que me ayudaron en mis primeros pasos: Schapa, Charly (ya dejá de aparecer che!), DFS. Aquellos con los que nos cargamos materias al hombro, Mati, Piter, Nico Rosner. La banda que hizo patria en Sistemas, Pablete, Mariano.

Los pibes del FEM!, en especial Marce con quien siempre compartimos interminables discusiones sobre la ciencia, la política científica y la historia universitaria. Manu y Nico, Lau, Lipa y Nano, Mauro, la primera camada. La CoNEAU ¿se acuerdan de la CoNEAU?

Los fieles de la bondiola. Dani V., Fran, Facu, Mica, Fede, Marianito y tantos otros de los que me estoy olvidando. Interminables comidas bajo la incumplible consigna de ``hoy no hablamos del DC''.

Toda la gente que ocupa, física o espiritualmente, la oficina 8. Mariano, Marcelo, Manu, Cani, Pau, Nico, Charly ¡Limpien el mate cuando se van!

Ya vengo llegando al final. Mis últimos recuerdos son para mis directores, Nico y Charly. Qué veranito ¡eh!

Al final, no sé en cuál de todos estos recuerdos está el punto de inflexión que me llevó hasta acá. Creo que fue un poco de todo. Lo que es seguro es que si uno de ustedes hubiera faltado, yo no habría sido el mismo, mi trabajo no habría sido el mismo. Por eso, a todos ustedes, muchas gracias. Más que muchas, infinitas.